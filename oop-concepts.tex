\documentclass{article}


\usepackage[]{geometry}

\usepackage{titling} % center cover page
\renewcommand\maketitlehooka{\null\vfill} % center cover page
\renewcommand\maketitlehookd{\vfill\null} % center cover page

\usepackage{tocloft} % add dots for sections in table of contents
\renewcommand{\cftsecleader}{\cftdotfill{\cftdotsep}} % add dots for sections in table of contents

\usepackage{hyperref} % make table of contents clickable
\hypersetup{ % remove default ugly look of links
	colorlinks,
	citecolor=black,
	filecolor=black,
	linkcolor=black,
	urlcolor=black
}

%\AddToHook{cmd/section/before}{\clearpage} % each section starts on new page

\usepackage{acronym} % acronyms





\hypersetup{pdftitle={OOP Concepts - English}}
\hypersetup{pdfauthor={Spyros Alertas}}
\hypersetup{pdfsubject={OOP Concepts - Notes}}
\hypersetup{pdfkeywords={Object Oriented Programming, OOP, Class, Object, Inheritance, Encapsulation, Abstraction, Polymorphism, Abstract Class, Interface, Method Overloading, Method Overriding}}

\title{\bfseries{\ac{OOP} Concepts}}

\author{Spyros Alertas}

\date{Created On: July 22, 2023\\Last Edit: \today}

% Hint: \title{whatever}, \author{who cares} and \date{whenever} could stand before or after the \begin{document} command BUT the \maketitle command MUST come AFTER the \begin{document} command!

\begin{document}


% cover page
\begin{titlingpage}
\maketitle
\end{titlingpage}


% table of contents
\tableofcontents


% all sections - main document content
\newpage
\section{Introduction}
\paragraph{} This document is a brief introduction to the basic concepts/principles of \acf{OOP}.\\
\paragraph{} The most important \acs{OOP} concepts are
\begin{itemize}
	\item \hyperref[sec:class]{Class}
	\item \hyperref[sec:object]{Object}
	\item \hyperref[sec:inheritance]{Inheritance}
	\item \hyperref[sec:encapsulation]{Encapsulation}
	\item \hyperref[sec:abstraction]{Abstraction}
	\item \hyperref[sec:polymorphism]{Polymorphism}
\end{itemize}


\section{Procedural vs \acl{OOP}}
\paragraph{} In \textbf{Procedural Programming} the program is divided into procedures. Each procedure contains all the logical steps required to solve a specific problem/task.
\paragraph{} In \textbf{\acl{OOP}} the program is designed around Objects. Objects contain data and methods (procedures) that act on these data.


\section{Why \acl{OOP}?}
\paragraph{} \textbf{\acf{OOP}} is very useful to create large and complex applications.
\paragraph{Advantages of \acs{OOP}:}
\begin{itemize}
	\item Modularity
	\item Re-usability
	\item Readability
	\item Security
\end{itemize}
\paragraph{Disadvantages of \acs{OOP}:}
\begin{itemize}
	\item Performance Cost
	\item Memory Cost
	\item Steeper Learning Curve
	\item Design Complexity
\end{itemize}


\section{Class \& Object}
\label{sec:class}
\label{sec:object}
\paragraph{} A \textbf{Class} is a user defined data type that acts as blueprint and is used to create \textbf{Objects}. A class defines the data and the operations that are allowed on these data.
\paragraph{} \textbf{Objects} are instances of classes. Each time an instance of a class is created, all class fields are initialized and separate memory space is allocated for each instance.


\section{Inheritance}
\label{sec:inheritance}
\paragraph{} \textbf{Inheritance} is a mechanism in which one class (called \textbf{Child Class} or \textbf{Subclass}) acquires all the properties and behaviors of another class (called \textbf{Parent}, \textbf{Super} or \textbf{Base Class}).


\section{Encapsulation}
\label{sec:encapsulation}
\paragraph{} \textbf{Encapsulation} is the process of integrating the data and the code (methods) into a single unit (class). The data are hidden from other classes and can be directly accessed only by methods from the same class they are found.
\paragraph{} \textbf{Encapsulation} can be achieved by using access modifiers (\acs{e.g.}: private, public, etc).


\section{Abstraction}
\label{sec:abstraction}
\paragraph{} \textbf{Abstraction} is the process of hiding implementation details in order to reduce complexity and improve efficiency.
\paragraph{} \textbf{Abstraction} can be achieved through interfaces and abstract classes.


\section{Polymorphism (Compile-Time \& Runtime)}
\label{sec:polymorphism}
\paragraph{} \textbf{Polymorphism} is the provision of a single interface to entities of different types. The program has the ability to identify the correct type at each moment.
\paragraph{} \textbf{Compile-Time} or \textbf{Static} or \textbf{Early Binding Polymorphism} can be achieved through \hyperref[sec:method-overloading]{method overloading}.
\paragraph{} \textbf{Runtime} or \textbf{Dynamic} or \textbf{Late Binding Polymorphism} can be achieved through \hyperref[sec:method-overriding]{method overriding}.


\newpage
\section{Abstract Class \& Abstract Method}
\label{sec:abstract-class}
\label{sec:abstract-method}
\paragraph{} An \textbf{Abstract Class} is a type of class that has at least one abstract method.
\paragraph{} An \textbf{Abstract Class}:
\begin{itemize}
	\item cannot be instantiated (cannot be used to create objects)
	\item can have variables
	\item can have non-abstract (implemented) methods
	\item must have at least one abstract method
\end{itemize}
\paragraph{} An \textbf{Abstract Method} can exist only in an abstract class and contains only the definition/signature (return type, parameters and exception) of the method without the body/implementation.


\section{Interface}
\label{sec:interface}
\paragraph{} An \textbf{Interface} is used as a common protocol for all the classes that implement it, as these classes will have to implement all the methods that are defined in the interface.
\paragraph{} An \textbf{Interface}:
\begin{itemize}
	\item cannot be instantiated (cannot be used to create objects)
	\item can have only static final variables
	\item cannot have implemented methods
\end{itemize}


\section{Method Overloading \& Method Overriding}
\label{sec:method-overloading}
\label{sec:method-overriding}
\paragraph{} \textbf{Method Overloading} is when two or more methods share the same name but have different signature (number and/or type of parameters).
\paragraph{} \textbf{Method Overriding} happens between child and parent class or classes that child class implicitly inherits from when methods have the same signature (name and parameters)


\section{Instance \& Class Variables}
\paragraph{} \textbf{Instance Variables} are unique for each instance of the class. Instance variables can be accessed using the object reference, not the class name.
\paragraph{} \textbf{Class Variables} are common to all instances of the class. Class variables have the keyword static and can be accessed using the class name.


\newpage % useful external resources
\section{External Resources}
\begin{itemize}
\item\href{https://en.wikipedia.org/wiki/Procedural_programming}{Procedural Programming - Wikipedia}
\item\href{https://en.wikipedia.org/wiki/Object-oriented_programming}{\acl{OOP} - Wikipedia}
\item\href{https://en.wikipedia.org/wiki/Programming_paradigm}{Programming Paradigms - Wikipedia}
\end{itemize}


% last section of the document - contains the list of used acronyms
%\newpage % acronyms should be on a new page
\section{Acronyms}
\begin{acronym}
	\acro{OOP}{Object Oriented Programming}
	\acro{e.g.}{latin: exempli gratia - english: for example}
\end{acronym}


\end{document}
